\documentclass[conference,compsoc,ngerman]{IEEEtran}

% Some/most Computer Society conferences require the compsoc mode option,
% but others may want the standard conference format.
%
% If IEEEtran.cls has not been installed into the LaTeX system files,
% manually specify the path to it like:
% \documentclass[conference,compsoc]{../sty/IEEEtran}





% Some very useful LaTeX packages include:
% (uncomment the ones you want to load)


% *** MISC UTILITY PACKAGES ***
%
%\usepackage{ifpdf}
% Heiko Oberdiek's ifpdf.sty is very useful if you need conditional
% compilation based on whether the output is pdf or dvi.
% usage:
% \ifpdf
%   % pdf code
% \else
%   % dvi code
% \fi
% The latest version of ifpdf.sty can be obtained from:
% http://www.ctan.org/pkg/ifpdf
% Also, note that IEEEtran.cls V1.7 and later provides a builtin
% \ifCLASSINFOpdf conditional that works the same way.
% When switching from latex to pdflatex and vice-versa, the compiler may
% have to be run twice to clear warning/error messages.






% *** CITATION PACKAGES ***
%
\ifCLASSOPTIONcompsoc
  % IEEE Computer Society needs nocompress option
  % requires cite.sty v4.0 or later (November 2003)
  \usepackage[nocompress]{cite}
\else
  % normal IEEE
  \usepackage{cite}
\fi
% cite.sty was written by Donald Arseneau
% V1.6 and later of IEEEtran pre-defines the format of the cite.sty package
% \cite{} output to follow that of the IEEE. Loading the cite package will
% result in citation numbers being automatically sorted and properly
% "compressed/ranged". e.g., [1], [9], [2], [7], [5], [6] without using
% cite.sty will become [1], [2], [5]--[7], [9] using cite.sty. cite.sty's
% \cite will automatically add leading space, if needed. Use cite.sty's
% noadjust option (cite.sty V3.8 and later) if you want to turn this off
% such as if a citation ever needs to be enclosed in parenthesis.
% cite.sty is already installed on most LaTeX systems. Be sure and use
% version 5.0 (2009-03-20) and later if using hyperref.sty.
% The latest version can be obtained at:
% http://www.ctan.org/pkg/cite
% The documentation is contained in the cite.sty file itself.
%
% Note that some packages require special options to format as the Computer
% Society requires. In particular, Computer Society  papers do not use
% compressed citation ranges as is done in typical IEEE papers
% (e.g., [1]-[4]). Instead, they list every citation separately in order
% (e.g., [1], [2], [3], [4]). To get the latter we need to load the cite
% package with the nocompress option which is supported by cite.sty v4.0
% and later.





% *** GRAPHICS RELATED PACKAGES ***
%
\ifCLASSINFOpdf
  % \usepackage[pdftex]{graphicx}
  % declare the path(s) where your graphic files are
  % \graphicspath{{../pdf/}{../jpeg/}}
  % and their extensions so you won't have to specify these with
  % every instance of \includegraphics
  % \DeclareGraphicsExtensions{.pdf,.jpeg,.png}
\else
  % or other class option (dvipsone, dvipdf, if not using dvips). graphicx
  % will default to the driver specified in the system graphics.cfg if no
  % driver is specified.
  % \usepackage[dvips]{graphicx}
  % declare the path(s) where your graphic files are
  % \graphicspath{{../eps/}}
  % and their extensions so you won't have to specify these with
  % every instance of \includegraphics
  % \DeclareGraphicsExtensions{.eps}
\fi
% graphicx was written by David Carlisle and Sebastian Rahtz. It is
% required if you want graphics, photos, etc. graphicx.sty is already
% installed on most LaTeX systems. The latest version and documentation
% can be obtained at: 
% http://www.ctan.org/pkg/graphicx
% Another good source of documentation is "Using Imported Graphics in
% LaTeX2e" by Keith Reckdahl which can be found at:
% http://www.ctan.org/pkg/epslatex
%
% latex, and pdflatex in dvi mode, support graphics in encapsulated
% postscript (.eps) format. pdflatex in pdf mode supports graphics
% in .pdf, .jpeg, .png and .mps (metapost) formats. Users should ensure
% that all non-photo figures use a vector format (.eps, .pdf, .mps) and
% not a bitmapped formats (.jpeg, .png). The IEEE frowns on bitmapped formats
% which can result in "jaggedy"/blurry rendering of lines and letters as
% well as large increases in file sizes.
%
% You can find documentation about the pdfTeX application at:
% http://www.tug.org/applications/pdftex





% *** MATH PACKAGES ***
%
%\usepackage{amsmath}
% A popular package from the American Mathematical Society that provides
% many useful and powerful commands for dealing with mathematics.
%
% Note that the amsmath package sets \interdisplaylinepenalty to 10000
% thus preventing page breaks from occurring within multiline equations. Use:
%\interdisplaylinepenalty=2500
% after loading amsmath to restore such page breaks as IEEEtran.cls normally
% does. amsmath.sty is already installed on most LaTeX systems. The latest
% version and documentation can be obtained at:
% http://www.ctan.org/pkg/amsmath





% *** SPECIALIZED LIST PACKAGES ***
%
%\usepackage{algorithmic}
% algorithmic.sty was written by Peter Williams and Rogerio Brito.
% This package provides an algorithmic environment fo describing algorithms.
% You can use the algorithmic environment in-text or within a figure
% environment to provide for a floating algorithm. Do NOT use the algorithm
% floating environment provided by algorithm.sty (by the same authors) or
% algorithm2e.sty (by Christophe Fiorio) as the IEEE does not use dedicated
% algorithm float types and packages that provide these will not provide
% correct IEEE style captions. The latest version and documentation of
% algorithmic.sty can be obtained at:
% http://www.ctan.org/pkg/algorithms
% Also of interest may be the (relatively newer and more customizable)
% algorithmicx.sty package by Szasz Janos:
% http://www.ctan.org/pkg/algorithmicx




% *** ALIGNMENT PACKAGES ***
%
%\usepackage{array}
% Frank Mittelbach's and David Carlisle's array.sty patches and improves
% the standard LaTeX2e array and tabular environments to provide better
% appearance and additional user controls. As the default LaTeX2e table
% generation code is lacking to the point of almost being broken with
% respect to the quality of the end results, all users are strongly
% advised to use an enhanced (at the very least that provided by array.sty)
% set of table tools. array.sty is already installed on most systems. The
% latest version and documentation can be obtained at:
% http://www.ctan.org/pkg/array


% IEEEtran contains the IEEEeqnarray family of commands that can be used to
% generate multiline equations as well as matrices, tables, etc., of high
% quality.




% *** SUBFIGURE PACKAGES ***
%\ifCLASSOPTIONcompsoc
%  \usepackage[caption=false,font=footnotesize,labelfont=sf,textfont=sf]{subfig}
%\else
%  \usepackage[caption=false,font=footnotesize]{subfig}
%\fi
% subfig.sty, written by Steven Douglas Cochran, is the modern replacement
% for subfigure.sty, the latter of which is no longer maintained and is
% incompatible with some LaTeX packages including fixltx2e. However,
% subfig.sty requires and automatically loads Axel Sommerfeldt's caption.sty
% which will override IEEEtran.cls' handling of captions and this will result
% in non-IEEE style figure/table captions. To prevent this problem, be sure
% and invoke subfig.sty's "caption=false" package option (available since
% subfig.sty version 1.3, 2005/06/28) as this is will preserve IEEEtran.cls
% handling of captions.
% Note that the Computer Society format requires a sans serif font rather
% than the serif font used in traditional IEEE formatting and thus the need
% to invoke different subfig.sty package options depending on whether
% compsoc mode has been enabled.
%
% The latest version and documentation of subfig.sty can be obtained at:
% http://www.ctan.org/pkg/subfig




% *** FLOAT PACKAGES ***
%
%\usepackage{fixltx2e}
% fixltx2e, the successor to the earlier fix2col.sty, was written by
% Frank Mittelbach and David Carlisle. This package corrects a few problems
% in the LaTeX2e kernel, the most notable of which is that in current
% LaTeX2e releases, the ordering of single and double column floats is not
% guaranteed to be preserved. Thus, an unpatched LaTeX2e can allow a
% single column figure to be placed prior to an earlier double column
% figure.
% Be aware that LaTeX2e kernels dated 2015 and later have fixltx2e.sty's
% corrections already built into the system in which case a warning will
% be issued if an attempt is made to load fixltx2e.sty as it is no longer
% needed.
% The latest version and documentation can be found at:
% http://www.ctan.org/pkg/fixltx2e


%\usepackage{stfloats}
% stfloats.sty was written by Sigitas Tolusis. This package gives LaTeX2e
% the ability to do double column floats at the bottom of the page as well
% as the top. (e.g., "\begin{figure*}[!b]" is not normally possible in
% LaTeX2e). It also provides a command:
%\fnbelowfloat
% to enable the placement of footnotes below bottom floats (the standard
% LaTeX2e kernel puts them above bottom floats). This is an invasive package
% which rewrites many portions of the LaTeX2e float routines. It may not work
% with other packages that modify the LaTeX2e float routines. The latest
% version and documentation can be obtained at:
% http://www.ctan.org/pkg/stfloats
% Do not use the stfloats baselinefloat ability as the IEEE does not allow
% \baselineskip to stretch. Authors submitting work to the IEEE should note
% that the IEEE rarely uses double column equations and that authors should try
% to avoid such use. Do not be tempted to use the cuted.sty or midfloat.sty
% packages (also by Sigitas Tolusis) as the IEEE does not format its papers in
% such ways.
% Do not attempt to use stfloats with fixltx2e as they are incompatible.
% Instead, use Morten Hogholm'a dblfloatfix which combines the features
% of both fixltx2e and stfloats:
%
% \usepackage{dblfloatfix}
% The latest version can be found at:
% http://www.ctan.org/pkg/dblfloatfix




% *** PDF, URL AND HYPERLINK PACKAGES ***
%
%\usepackage{url}
% url.sty was written by Donald Arseneau. It provides better support for
% handling and breaking URLs. url.sty is already installed on most LaTeX
% systems. The latest version and documentation can be obtained at:
% http://www.ctan.org/pkg/url
% Basically, \url{my_url_here}.




% *** Do not adjust lengths that control margins, column widths, etc. ***
% *** Do not use packages that alter fonts (such as pslatex).         ***
% There should be no need to do such things with IEEEtran.cls V1.6 and later.
% (Unless specifically asked to do so by the journal or conference you plan
% to submit to, of course. )


% correct bad hyphenation here
\hyphenation{op-tical net-works semi-conduc-tor}

% URL benutzen
\usepackage{hyperref}
% Deutsche Bezeichnungen
\usepackage[main=ngerman, english]{babel}
\usepackage{cleveref}
\usepackage[left=2cm,right=2cm,top=2.4cm,bottom=2.4cm]{geometry}

\usepackage[pdftex]{graphicx}
% declare the path(s) where your graphic files are
\graphicspath{{../pdf/}{../jpeg/}}
% and their extensions so you won't have to specify these with
% every instance of \includegraphics
\DeclareGraphicsExtensions{.pdf,.jpeg,.png}
\usepackage[export]{adjustbox}

\begin{document}
%
% paper title
% Titles are generally capitalized except for words such as a, an, and, as,
% at, but, by, for, in, nor, of, on, or, the, to and up, which are usually
% not capitalized unless they are the first or last word of the title.
% Linebreaks \\ can be used within to get better formatting as desired.
% Do not put math or special symbols in the title.
\title{Kontextabhängige Choreographie-Anpassung: Eine Literaturrecherche}


% author names and affiliations
% use a multiple column layout for up to three different
% affiliations
\author{\IEEEauthorblockN{Senait Behre}
\IEEEauthorblockA{Matrikelnummer: 2670344\\
inf80688@stud.uni-stuttgart.de \\
Universität Stuttgart}
\and
\IEEEauthorblockN{Markus Schütterle}
\IEEEauthorblockA{Matrikelnummer: 2752262 \\
st154063@stud.uni-stuttgart.de\\
Universität Stuttgart}
\and
\IEEEauthorblockN{Marcel Zeller}
\IEEEauthorblockA{Matrikelnummer: 3294457\\
st154091@stud.uni-stuttgart.de\\	
Universität Stuttgart}}

% conference papers do not typically use \thanks and this command
% is locked out in conference mode. If really needed, such as for
% the acknowledgment of grants, issue a \IEEEoverridecommandlockouts
% after \documentclass

% for over three affiliations, or if they all won't fit within the width
% of the page (and note that there is less available width in this regard for
% compsoc conferences compared to traditional conferences), use this
% alternative format:
% 
%\author{\IEEEauthorblockN{Michael Shell\IEEEauthorrefmark{1},
%Homer Simpson\IEEEauthorrefmark{2},
%James Kirk\IEEEauthorrefmark{3}, 
%Montgomery Scott\IEEEauthorrefmark{3} and
%Eldon Tyrell\IEEEauthorrefmark{4}}
%\IEEEauthorblockA{\IEEEauthorrefmark{1}School of Electrical and Computer Engineering\\
%Georgia Institute of Technology,
%Atlanta, Georgia 30332--0250\\ Email: see http://www.michaelshell.org/contact.html}
%\IEEEauthorblockA{\IEEEauthorrefmark{2}Twentieth Century Fox, Springfield, USA\\
%Email: homer@thesimpsons.com}
%\IEEEauthorblockA{\IEEEauthorrefmark{3}Starfleet Academy, San Francisco, California 96678-2391\\
%Telephone: (800) 555--1212, Fax: (888) 555--1212}
%\IEEEauthorblockA{\IEEEauthorrefmark{4}Tyrell Inc., 123 Replicant Street, Los Angeles, California 90210--4321}}




% use for special paper notices
%\IEEEspecialpapernotice{(Invited Paper)}




% make the title area
\maketitle

% As a general rule, do not put math, special symbols or citations
% in the abstract
\begin{abstract}
Eine Service-Choreographie sind mehrere Services, welche sich dezentral organisieren, indem die einzelnen Services miteinander kommunizieren. Um sich ändernden Anforderungen anzupassen und die Servicequalität zu erhöhen werden Strategien zur Anpassung von Service-Choreographien benötigt.
Das Ziel dieser systematischen Literaturrecherche ist es den aktuellen Forschungsstand zu kontextabhängiger Anpassung von Service-Choreographien zusammenzufassen. Dabei wird das Vorgehen protokolliert, was die Reproduzierbarkeit dieser Literaturrecherche erhöht. Dabei wurden die Suchanfragen automatisiert auf mehreren Suchmaschinen ausgeführt. Insgesamt wurden 22
themenrelevante Ergebnisse gefunden. Dabei wurde für jedes Ergebnis zusammengefasst, welche Adaptionsstrategie verwendet wird. Außerdem wurde zusammengefasst, wie die Adaption durchgeführt wird, wobei auf Ziel, erforderlichen Eingriffsgrad, Kontextsensitivität, Auswirkungen auf die Skalierbarkeit, bestehende Implementierungen und zugrundeliegende Modelle eingegangen wird. Zudem werden bestehende Limitationen der Ansätze zusammengefasst.
\end{abstract}

% no keywords




% For peer review papers, you can put extra information on the cover
% page as needed:
% \ifCLASSOPTIONpeerreview
% \begin{center} \bfseries EDICS Category: 3-BBND \end{center}
% \fi
%
% For peerreview papers, this IEEEtran command inserts a page break and
% creates the second title. It will be ignored for other modes.
\IEEEpeerreviewmaketitle



\section{Einleitung}
Der Informationsgehalt einer Situation oder Handlung hängt entscheidend davon ab, ob der Kontext bekannt ist. Der Begriff Kontext fasst dabei alles zusammen, was in irgendeinem Zusammenhang mit der Situation oder Handlung steht. Das können Objekte, andere Handlungen oder Verhalten in der Umgebung sein. Aber auch der Ort, das Wetter, die Jahres- und Uhrzeit, die an der Situation Beteiligten, deren Gefühle und emotionale Situation, Gesichtsausdruck, Bewegungen und Tonfall können zum Kontext gehören. Diese Liste ist praktisch unbegrenzt erweiterbar. Handelt es sich bei der betrachteten Situation um das Ausführen einer Softwareapplikation, kann der Kontext neben Ort, Wetter, Zeit zum Beispiel auch aus dem Nutzer, Einstellungen oder Eigenschaften von Konnektivitätsschnittstellen bestehen. Während es ein Mensch vermag, auf den Kontext einer Situation angemessen zu reagieren, fällt dies einem Computer sehr schwer. Sehr viele Kombinationen und Varianten des Kontextes müssen von der kontext-sensitiven Anwendung berücksichtigt werden. Gleichzeitig verspricht dies aber auch eine erheblich verbesserte Servicequalität für die Nutzer.

Eine Service-Choreographie ist eine globale Beschreibung der teilnehmenden Services, die durch den Austausch von Nachrichten, Interaktionsregeln und Vereinbarungen zwischen zwei oder mehr Endpunkten definiert wird. Eine Choreographie verwendet einen dezentralisierten Ansatz für die Zusammenstellung von Dienstleistungen. Das bedeutet, dass die Services sich untereinander selbst organisieren. In einer solchen Service-Choreographie können die genutzten Services basierend auf Kontextinformationen ausgetauscht werden. Zum Beispiel könnte eine andere App für den öffentlichen Nahverkehr geladen werden, je nachdem in welchem Land sich ein Smartphone befindet. Gleichzeitig bleibt aber die von dieser App genutzte Bezahl-App für die Buchung eines Tickets dieselbe, da es sich um die App eines internationalen Finanzdienstleisters handelt.

Ziel dieser Arbeit ist die Identifizierung und Analyse des aktuellen Stands der Forschung in der kontextsensitiven Anpassung von Service-Choreographien. Die Identifikation der dazu relevanten Literatur erfolgt durch eine systematische Literaturrecherche \cite{budgen2006performing,kitchenham2009systematic}. Anders als bei der üblichen Literaturrecherche verringert die systematische Literaturrecherche die Verzerrung~\cite{keele2007guidelines}. Die systematische Literaturrecherche folgt im Gegensatz zur normalen Literaturrecherche einer definierten Abfolge von methodischen Schritten. Dabei wird das Vorgehen genau protokolliert, damit die Literaturrecherche für andere Forscher nachvollziehbar und reproduzierbar ist.

Diese Arbeit gliedert sich wie folgt. In \cref{s:forschungsmethode} wird das Protokoll dieser systematischen Literaturrecherche beschrieben. Anschließend werden in \cref{s:contextAwareResults} die Ergebnisse präsentiert. In \cref{s:verwandteArbeiten} werden verwandte Arbeiten diskutiert und die Arbeit endet in \cref{s:zusammenfassung} mit einer Zusammenfassung und einem Ausblick.




% An example of a floating figure using the graphicx package.
% Note that \label must occur AFTER (or within) \caption.
% For figures, \caption should occur after the \includegraphics.
% Note that IEEEtran v1.7 and later has special internal code that
% is designed to preserve the operation of \label within \caption
% even when the captionsoff option is in effect. However, because
% of issues like this, it may be the safest practice to put all your
% \label just after \caption rather than within \caption{}.
%
% Reminder: the "draftcls" or "draftclsnofoot", not "draft", class
% option should be used if it is desired that the figures are to be
% displayed while in draft mode.
%
%\begin{figure}[!t]
%\centering
%\includegraphics[width=2.5in]{myfigure}
% where an .eps filename suffix will be assumed under latex, 
% and a .pdf suffix will be assumed for pdflatex; or what has been declared
% via \DeclareGraphicsExtensions.
%\caption{Simulation results for the network.}
%\label{fig_sim}
%\end{figure}

% Note that the IEEE typically puts floats only at the top, even when this
% results in a large percentage of a column being occupied by floats.


% An example of a double column floating figure using two subfigures.
% (The subfig.sty package must be loaded for this to work.)
% The subfigure \label commands are set within each subfloat command,
% and the \label for the overall figure must come after \caption.
% \hfil is used as a separator to get equal spacing.
% Watch out that the combined width of all the subfigures on a 
% line do not exceed the text width or a line break will occur.
%
%\begin{figure*}[!t]
%\centering
%\subfloat[Case I]{\includegraphics[width=2.5in]{box}%
%\label{fig_first_case}}
%\hfil
%\subfloat[Case II]{\includegraphics[width=2.5in]{box}%
%\label{fig_second_case}}
%\caption{Simulation results for the network.}
%\label{fig_sim}
%\end{figure*}
%
% Note that often IEEE papers with subfigures do not employ subfigure
% captions (using the optional argument to \subfloat[]), but instead will
% reference/describe all of them (a), (b), etc., within the main caption.
% Be aware that for subfig.sty to generate the (a), (b), etc., subfigure
% labels, the optional argument to \subfloat must be present. If a
% subcaption is not desired, just leave its contents blank,
% e.g., \subfloat[].


% An example of a floating table. Note that, for IEEE style tables, the
% \caption command should come BEFORE the table and, given that table
% captions serve much like titles, are usually capitalized except for words
% such as a, an, and, as, at, but, by, for, in, nor, of, on, or, the, to
% and up, which are usually not capitalized unless they are the first or
% last word of the caption. Table text will default to \footnotesize as
% the IEEE normally uses this smaller font for tables.
% The \label must come after \caption as always.
%
%\begin{table}[!t]
%% increase table row spacing, adjust to taste
%\renewcommand{\arraystretch}{1.3}
% if using array.sty, it might be a good idea to tweak the value of
% \extrarowheight as needed to properly center the text within the cells
%\caption{An Example of a Table}
%\label{table_example}
%\centering
%% Some packages, such as MDW tools, offer better commands for making tables
%% than the plain LaTeX2e tabular which is used here.
%\begin{tabular}{|c||c|}
%\hline
%One & Two\\
%\hline
%Three & Four\\
%\hline
%\end{tabular}
%\end{table}


% Note that the IEEE does not put floats in the very first column
% - or typically anywhere on the first page for that matter. Also,
% in-text middle ("here") positioning is typically not used, but it
% is allowed and encouraged for Computer Society conferences (but
% not Computer Society journals). Most IEEE journals/conferences use
% top floats exclusively. 
% Note that, LaTeX2e, unlike IEEE journals/conferences, places
% footnotes above bottom floats. This can be corrected via the
% \fnbelowfloat command of the stfloats package.




\section{Forschungsmethode}\label{s:forschungsmethode}
Um den aktuellen Forschungsstand zu erfassen und zu evaluieren wird die systematische Literaturrecherche (SLR) nach Kitchenham~\cite{kitchenham2004evidence,keele2007guidelines} verwendet. Die Methode dient dazu, den aktuell verfügbaren Forschungsstand zu einem bestimmten Thema zu identifizieren und zusammenzufassen. Dabei ist eine klare Methodik vorgegeben was die Ergebnisse weniger verzerrt~\cite{keele2007guidelines}.
Das Protokoll zur systematischen Literaturrecherche umfasst alle notwendigen Informationen, wie die Fragestellungen, Suchanfragen oder die Kriterien zur Auswahl der Ergebnisse. Durch das Protokoll ist die Recherche wiederholbar, sodass sich die Ergebnisse jederzeit reproduzieren lassen.

Das Protokoll der systematischen Literaturrecherche wurde auf Basis der Richtlinien von Kitchenham und Charters~\cite{keele2007guidelines} entwickelt und durchgeführt. Für den Aufbau des Protokolls dienten die Protokolle von Kitchenham~\cite{kitchenham2009systematic}, Leite et al.~\cite{leite2013systematic} und Dyba~\cite{dybaa2008empirical} als Vorbild.
\subsection{Forschungsfragen}\label{ss:forschungsfragen}
Bei der Analyse und Zusammenfassung der Ergebnisse wird versucht vorher definierte Forschungsfragen zu beantworten, um die Ergebnisse vergleichbar darzustellen.
Die Forschungsfragen sind an den Fragen der Literaturrecherche von Leite et al.~\cite{leite2013systematic} orientiert, da diese sich mit Strategien zur Anpassung von Service-Choreograpien befasst. Die Forschungsfragen RQ1 und RQ2 sind weitgehend von Leite et al.\cite{leite2013systematic} übernommen, wobei die Fragen auf die Thematik der vorliegenden SLR angepasst wurden, sodass auch auf die Kontextabhängigkeit eingegangen wird.
Für diese systematische Literaturrecherche wurden die folgenden Forschungsfragen verwendet:


\textbf{RQ1}: Welche Strategie zur Anpassung benutzt die ausgewählte Studie für die kontextabhängige Choreographie-Anpassung? 

\textbf{RQ2}: Wie wählt eine ausgewählte Studie ihre Adaptionsstrategie gemäß den folgenden Aspekten aus?

RQ2.1 \textit{Ziel}: Unterstützt die kontextabhänige Anpassung der Choreographie funktionale oder nicht funktionale Anforderungen?
%Funktional beschreiben gewünschte Funktionalitäten
% Nichtfunktionale Sind Anforderungen an die Qualität welche die geforderte Funktionalität erbringen muss

RQ2.2 \textit{Erforderlicher Eingriffsgrad}: Wird die kontextabhänige Anpassung der Choreographie automatisch durchgeführt oder ist menschliches Eingreifen notwendig?

RQ2.3 \textit{Kontextsensitivität}: Welchen Kontext bezieht die Choreographie-Anpassungs-Strategie in ihre Entscheidungsfindung mit ein? Wie stark wird die Entscheidungsfindung durch den Kontext beeinflusst?

RQ2.4 \textit{Auswirkungen auf die Skalierbarkeit}: Wird der Einfluss auf die Skalierbarkeit der Choreographie diskutiert? Falls ja, ist es informell oder sind formale Beweise oder Experimente enthalten?

RQ2.5 \textit{Implementierungen}: Wird der vorgestellte Ansatz von einem Tool oder Prototyp implementiert und ist die Implementierung des Systems verfügbar? Falls ja, ist es Open-Source-Software?

RQ2.6 \textit{Zugrundeliegende Modelle}: Welche Modelle oder Standards werden in dem Ansatz verwendet?

\textbf{RQ3}: Was sind die Hauptlimitationen und offenen Forschungsfragen der Ansätze von \glqq kontextabhängiger Choreographie-Anpassung \grqq?

%Alternative Fragen:
%
%(i) Gibt es ein System, dass auf Situationen reagiert ohne Einfluss eines Menschen?
%
%(ii) Was sind die Vor- und Nachteile eines adaptiven Systems
%
%(iii) Welche Methoden gibt es, um ein System vom Kontext abhängig zu machen?
%
%(iv) Welche Grenzen werden einem Situationsabhängigen System gegeben?


\subsection{Datenquellen}\label{ss:datenquellen}
Die Ergebnisse wurden aus verschiedenen öffentlichen Datenquellen extrahiert. Als Datenquellen für die Suche wurden die folgenden wissenschaftlichen Quellen mit einer definierten Zeichenabfolge (siehe \cref{section:suchanfrage}) durchsucht: IEEE Xplore\footnote{\url{https://ieeexplore.ieee.org/Xplore/home.jsp}} SpringerLink\footnote{\url{https://link.springer.com/}} und Google Scholar\footnote{\url{https://scholar.google.de/}}

Um die Reproduzierbarkeit der durchgeführten systematischen Literaturrecherche zu gewährleisten, wurde der Suchprozess durch Skripte automatisiert. Dazu wurden die
genannten Datenquellen mit einem Python Skript durchsucht \footnote{\url{https://github.com/Lenny925/Context-Aware-Choreography-Adaption}}.


\subsection{Abfragezeichenfolge}
\label{section:suchanfrage}
Die Suchen auf den Datenquellen (siehe \cref{ss:datenquellen}) wurden mithilfe von Suchanfragen durchgeführt. Dabei wurden Titel und Zusammenfassung der gefundenen Ergebnisse durchsucht.
Die Suchanfragen sind an den Suchanfragen der systematischen Literaturrecherche von Leite et al.~\cite{leite2013systematic} orientiert und wurden um Suchwörter zu \textit{kontextabhängig} erweitert.
Aus der Kombination entstanden 3 Suchblöcke. Der erste Block ist für \textit{kontextabhängig}, der zweite für \textit{Choreographie} und der dritte Block für \textit{Anpassung}. Die Suchwörter aus dem zweiten und dritten Block sind aus den Suchanfragen der Literaturrecherche von Leite et al.~\cite{leite2013systematic}. Die verschiedenen Zeichenfolgen aus einer Gruppe sind mit ODER (OR) verbunden. Die Suchblöcke werden mit UND (AND) verbunden. Dadurch entstehen mehrere Anfragen wobei aus jedem Block eine Zeichenfolge enthalten sein muss. Daraus wurden sämtliche Möglichkeiten gebildet und auf die jeweiligen Datenquellen (siehe \cref{ss:datenquellen}) ausgeführt. 
Für diese Literaturrecherche wurden die folgenden Suchanfragen verwendet:\\ \\
(
%- Context-aware computing\\
	\glqq context-aware\grqq~OR\\
%- Context-specific computing\\
	\glqq context-specific\grqq~OR\\
%- Context-dependent computing\\
	\glqq context-dependent\grqq~OR\\
%- Context-sensitive computing\\
	\glqq context-sensitiv\grqq~OR\\
%- Situation-aware computing\\
	\glqq situation-aware\grqq 
%"location" OR\\
%"situational" OR\\
%"environment" OR
)\\
AND\\
ab hier vgl. Leite et al.~\cite{leite2013systematic} \\
 (
\glqq choreography\grqq~OR\\
\glqq decentralized composition\grqq~OR\\
\glqq decentralized service composition\grqq~OR\\
\glqq distributed composition\grqq
) \\
AND\\
(
\glqq adapt*\grqq~OR\\
\glqq self-config*\grqq~OR\\
\glqq auto-config*\grqq~OR\\
\glqq reconfig*\grqq~OR\\
\glqq sensitiv\grqq~OR\\
\glqq behaviour\grqq
)\\

\subsection{Einschluss und Ausschluss Kriterien}
Da nicht jedes Ergebnis zum Thema passt, werden die Ergebnisse der Suche manuell überprüft. Dabei werden nur Paper in die Ergebnisse der systematischen Literaturrecherche aufgenommen, die sich mit der kontextabhängigen Anpassung von Service-Choreografien befassen. Dabei werden Ergebnisse ausgeschlossen, die keinen Kontext in die Anpassungsstrategie einbeziehen, oder für Orchestrationen sind. Die kontextabhängige Anpassung einzelner Services wird auch ausgeschlossen, da der Fokus auf Choreografien mehrerer Services liegt.

Die gefundenen Ergebnisse durchlaufen, wie in Tabelle \ref{tabelle:auswahlverfahren} gezeigt, mehrere Schritte, wobei passende Ergebnisse ausgewählt werden. Nach der automatisierten Suche waren nach Entfernen der Duplikate insgesamt 4435 Ergebnisse vorhanden. Die Duplikate wurde mithilfe von Microsoft Excel über den Titel der Ergebnisse aussortiert.

Nach dem Entfernen der Duplikate wurden die Ergebnisse auf Basis des Veröffentlichungsjahres und nach der Anzahl der Zitationen vorgefiltert.
Alle Paper vor dem Jahr 2000 wurden ausgeschlossen. Für die Jahre 2000-2010 wurden mindestens 10 Zitationen als Einschlusskriterium verwendet. Für das Jahr 2011 wurden mindestens 9 Zitationen, für das Jahr 2012 mindestens 8 Zitationen, für das Jahr 2013 mindestens 7 Zitationen, für das Jahr 2014 mindestens 6 Zitationen, für das Jahr 2015 mindestens 5 Zitationen, für das Jahr 2016 mindestens 4 Zitationen, für das Jahr 2017 mindestens 3 Zitationen, für das Jahr 2018 mindestens 2 Zitationen und für das Jahr 2019 mindestens 1 Ziation vorausgesetzt. Als Ergebnis blieben 2722 Ergebnisse übrig.

Da nicht jedes Ergebnis themenrelevant ist, wurden die verbleibenden Ergebnisse manuell weiter sortiert. Dazu wurden die zu Anfang dieses Abschnittes festgelegte Einschluss- und Ausschlusskriterien auf die Titel der Ergebnisse und im Zweifelsfall auf die Zusammenfassung angewendet. Als Resultat blieben 89 Ergebnisse übrig.

Bei den verbleibenden 89 Ergebnissen wurden die Einschluss- und Ausschlusskriterien auf den ganzen Text angewendet. Daraus resultieren die Ergebnisse der vorliegenden systematischen Literaturrecherche. Die Ergebnisse werden in \cref{s:contextAwareResults} zusammengefasst.

\begin{table}[h]
	\caption{Studienauswahlverfahren}
	\label{tabelle:auswahlverfahren}
\begin{tabular}{p{1cm}|p{6.5cm}}

	Stufe & Beschreibung \\
	\hline	
	Stufe 1 & Die Suchanfragen wurden auf allen angegeben Datenquellen angewendet und die Ergebnisse gespeichert.\\
	
	Stufe 2 & Duplikate und ungültige Paper wurden ausgeschlosssen.\\
	
	Stufe 3 & Die Inklusions- und Exklusionskriterien wurden auf die Titel der Paper und bei Unklarheit auf den Abstract angewendet. \\
	
	%Stufe 4 & Die Inklusions- und Exklusionskriterien wurden auf Abstract und Zusammenfassung anwendet. \\
	
	Stufe 4 & Die Inklusions- und Exklusionskriterien wurden auf den ganzen Text angewendet.
	
\end{tabular}
\end{table}


\subsection{Datenextrahierung}
Bei den automatisierten Suchen werden verschiedene Metadaten über die Ergebnisse gesammelt. Die Daten wurden mithilfe des automatisierten Suchskriptes extrahiert.
Jede der Anfragen wurde einzeln auf jeder der genannten Suchmaschinen ausgeführt (siehe \cref{ss:datenquellen}). Die Ergebnisse der einzelnen Suchanfragen wurden jeweils in einer separaten CSV-Datei gespeichert, wobei für jedes Ergebnis die folgenden Metadaten extrahiert und gespeichert wurden:

\begin{itemize}
	\item Titel
	\item Autor(en)
	\item Veröffentlichungsjahr
	\item Link zum Ergebnis
	\item Link zum PDF
	\item Zugehörige Suchanfrage
	\item Datenquelle
	\item Zahl der Zitationen
	\item Digitale Objekt ID (DOI) des Ergebnisses
\end{itemize}

Die Ergebnisse der einzelnen Suchfragen wurden anschließend in einer einzelnen CSV-Datei zusammengefasst, wobei 9168 Paper als Ergebnis resultieren. Da die Suchanfragen auf verschiedenen Suchmaschinen ausgeführt wurden, sind in den Ergebnissen Duplikate enthalten. Die Duplikate wurden deshalb mithilfe von Microsoft Excel aus den Ergebnissen entfernt.
Nach dem Entfernen der Duplikate blieben 4435 Ergebnisse übrig.

\subsection{Protokoll Evaluierung und Limitierungen}
Die systematische Literaturrecherche wurde mit einem Python-Script\footnote{\url{https://github.com/Lenny925/Context-Aware-Choreography-Adaption}} automatisiert. Dazu wurden die Suchanfragen auf den verschiedenen Datenquellen ausgeführt und die Metadaten, wie Titel, Autor oder Anzahl der Zitationen heruntergeladen. Dabei waren nicht für jedes Ergebnis alle Metadaten vorhanden. Manche Ergebnisse haben zum Beispiel keine Zitationen oder keinen \textit{Digital Object Identifier} (DOI)

Bei Springer Link gibt es keine offizielle API für automatische Suchen. Die Anzahl der möglichen Anfragen ist nicht begrenzt und auch bei einer Vielzahl von Anfragen wird kein Bann gegen den Benutzer verhängt. Allerdings können maximal 1000 Seiten pro Suche von Springer Link abgefragt werden, weshalb der Crawler limitiert wurde. 

Google Scholar beschränkt die Anzahl der Anfragen. Dies führt bei automatischen Suchen, wie im Rahmen dieser Arbeit, zu einem temporären Bann des Benutzers. Der Bann wird nach dem Crawlen von ungefähr 40 Seiten aktiv. Der Crawler für Google Scholar wurde deswegen auf 10 Seiten pro Suchanfrage begrenzt. Pro Seite werden 10 Ergebnisse angezeigt. Folglich werden für jede Suchanfrage maximal 100 Ergebnisse gecrawlt. 

IEEE Xplore bietet eine API\footnote{\url{https://developer.ieee.org/docs/read/Searching_the_IEEE_Xplore_Metadata_API}} für automatisierte Suchanfragen an. Dabei ist die Anzahl der Zugriffe limitiert. Für die Suchanfragen dieser Literaturrecherche (siehe \cref{section:suchanfrage}) war diese Limitation allerdings kein Problem, da es für die Anfragen auf IEEEXplore nur wenige Treffer gab.

%\section{Charakterisierung der ausgewählten Studien}
%Hier kommen allgemeine Sachen über die gefundenen Paper hin. Dies gibt eine Übersicht über die gefundenen Paper. Siehe SLR von Leite\cite{leite2013systematic}.
%
%Von welcher Datenquelle kommen wieviele Paper?
%
%Wie wurden die Paper veröffentlicht?
%
%In welchen Jahren wurden sie veröffentlicht?
%
%In welche Kategorie kann man die Paper einordnen?

\section{Strategien für kontextabhänige Choreographie-Anpassung}\label{s:contextAwareResults}
In diesem Abschnitt werden die gefundenen Ergebnisse aufgeführt, wobei diese im Hinblick auf die Forschungsfragen zusammengefasst werden. Dabei wird zunächst  ein Überblick über die kontextsensitive Choreografie-Adaptionsstrategie gezeigt, der \textbf{RQ1} beantwortet. Anschließend wird die Adaptionsstrategie betrachtet, was \textbf{RQ2} beantwortet. Am Ende wird \textbf{RQ3} beantwortet und die Hauptlimitationen und offenen Forschungsfragen des jeweiligen Ansatzes aufgezeigt.
Werden Fragen nicht beantwortet bedeutet dies, dass in dem Paper darauf nicht eingegangen wurde.

In den folgenden Unterabschnitten werden die Ergebnisse dieser Literaturrecherche aufgeführt und es wird versucht die Forschungsfragen (siehe \cref{ss:forschungsfragen} für jedes Ergebnis zu beantworten.

\subsection{Automated Context-Aware Adaptation of Web Service Executions}
Narendra und Gundugola~\cite{narendra2006automated} zeigen einen Ansatz zur Anpassung von zusammengesetzten Web-Service Ausführungen, wobei funktionale Anforderungen angepasst werden (RQ2.1). Dabei zeigen sie, wie die Anpassungen mithilfe von Kontextontologien modelliert werden können. Zur automatischen Anpassung werden Kontextinformationen auf verschiedenen Ebenen der Web-Services identifiziert (RQ2.2). Diese werden in verschiedene Typen klassifiziert: C-Kontext für den zusammengesetzten Service, W-Kontext für den Web-Service Provider und I-Kontext für einzelne Service-Instanzen, die von den Web-Service Providern erstellt werden (RQ2.3)\cite{narendra2006automated}. Anpassungen in einem Web-Service werden immer dann benötigt, wenn sich die Spezifikation einer Komponente ändert~\cite{narendra2006automated}. Dies kann beispielsweise der Austausch eines Services durch einen anderen mit erweiterten Anforderungen sein.
Als Beispiel für die Workflow-Anpassung wird die Ausnahmebehandlung genannt, die sich mit Mechanismen zur Wiederherstellung des Zustands des Workflows im Falle eines Fehlers während der Workflow-Ausführung befasst \cite{narendra2006automated}. Dieser Fall tritt ein, wenn eine der Web-Service-Komponenten nicht ausgeführt wird~\cite{narendra2006automated}.
Zur automatischen Anpassung werden Sphären verwendet, welche eine Sammlung von Aufgaben repräsentieren, die entweder komplett oder überhaupt nicht ausgeführt werden~\cite{narendra2006automated}. Sphären werden mit einem Algorithmus identifiziert. Der Algorithmus zur Anpassung des Workflows, spiegelt die Rückwärtsbewegungen des Sphären-Bestimmungsalgorithmus wider \cite{narendra2006automated}.
Der Ansatz soll in Zukunft noch in einem Prototyp implementiert werden, um die Machbarkeit zu beweisen (RQ2.5). Außerdem soll die Strategie zur Anpassung weiter erweitert werden, um weiteren Kontext einzubeziehen (RQ3).

\subsection{Modeling of Context-Aware Self-Adaptive Applications in Ubiquitous and Service-Oriented Environments}
Der Ansatz von Geihs et al.~\cite{geihs2009modeling} zielt auf Umgebungen mit dynamisch verfügbaren Diensten ab, die von Anwendungen genutzt werden können. Dabei ist das Ziel die Anpassung der funktionalen Anforderungen (RQ2.1). Die Anpassungsentscheidungen hängen dabei nicht nur von den Kontext-Eigenschaften ab, sondern auch von der Verfügbarkeit der Dienste (RQ2.3). Um Dienste dynamisch anzupassen, muss der Adaptionsansatz verschiedene Anforderungen erfüllen. Dazu gehören die Variabilität der Anwendung, eine dynamische Serviceerkennung, Heterogenität und die Integration von Service- und Kontext-Eigenschaften in die Planung~\cite{geihs2009modeling}.
Zur Adaption werden bei einem signifikanten Kontextwechsel alle verfügbaren Anwendungsvarianten von der Middleware ausgewertet und auf Basis verschiedener QoS-Metadaten verglichen, die den beteiligten Komponenten zugeordnet sind~\cite{geihs2009modeling}. Komponententypen sind Variationspunkte, die einer bestimmten Teilefunktionalität der Anwendung entsprechen. Die Komponententypen werden rekursiv ausgewählt und der Prozess wird gestoppt, wenn ein atomarer Realisierungsplan ausgewählt wurde.
Wenn Dienste für die Middleware nicht mehr verfügbar sind, wird der Realisierungsplan verworfen und ein neuer Anpassungsprozess wird gestartet~\cite{geihs2009modeling}.
Der Ansatz ist bereits teilweise implementiert und wurde verwendet, um erste Prototypen zu implementieren (RQ2.5). Weitere Forschungsmöglichkeiten gibt es im Bezug auf Performance und Skalierbarkeit (RQ3).

\subsection{Semantic Web Service Adaptation Model for a Pervasive Learning Scenario}
Lau et al.\cite{lau2008semantic} liefern einen halbautomatischen Adaptionsansatz zur Abstimmung und Anpassung von relevanten Web-Services (RQ2.2). Das Ziel ist die Anpassung von funktionalen und nicht funktionalen Anforderungen (RQ2.1). Um einen Service anzupassen oder durch einen anderen zu ersetzen wird eine \textit{Service Requirement Specification} verwendet, in welcher beschrieben wird, wie ein Service auszusehen hat. Außerdem werden Dienste mit einem sogenannten \textit{Service Descriptor} beschrieben. Die Anpassung besteht aus zwei Schritten die hintereinander ablaufen. Zuerst werden relevante Dienste basierend auf der Beschreibung des Services aus einer Datenbank gesucht, um die Serviceanforderungen für eine Aufgabe zu erfüllen. Die relevanten Services dienen als Input für die folgende Adaptionsphase, die dazu dient die relevanten Dienste an die aktuelle Situation und den Interessen der Nutzer anzupassen (RQ2.3). Der Adaptionsprozess besteht aus den drei Phasen Bewertung/Klassifizierung, Filterung und Ranking ~\cite{lau2008semantic}. Bei der Klassifizierung werden die Services entsprechend der aktuellen Situation und Klassen eingeteilt. In der Phase des Filterns werden irrelevante Dienste über die Klassifizierung herausgefiltert. Anschließend wird basierend auf ausgewählten Merkmalen ein Ranking erstellt, wobei die Services in eine Rangfolge gebracht werden.
Der Ansatz wurde in einem Prototyp implementiert, um die Funktionalität zu validieren (RQ2.5).

\subsection{A Conceptual Model for Adaptable Context-aware Services}
Autili et al.~\cite{autili2006conceptual} schlagen einen Ansatz für anpassbare kontext- und serviceorientierte Anwendungen vor. Dieser basiert auf einer 2-Layer-Architektur, welche aus einem Service-Layer und einem Komponenten-Layer besteht. Der Komponenten-Layer bildet die Rechenressourcen für die vernetzen Dienste ab, welche auf dem Serice-Layer abgebildet werden. Softwarekomponenten können zu jeder Zeit bereitgestellt, aktualisiert oder entfernt werden, ohne dabei die Funktionalität zu beeinflussen (RQ2.2). Das Mapping zwischen Komponenten und Services wird auf Service-Layer Ebene realisiert.
Die Adaption der Services soll funktionale und nicht funktionale Anforderungen verbessern, wobei das Ziel ist, einen optimalen Kompromiss zwischen Bedürfnissen der Nutzer und den aktuellen verfügbaren Ressourcen zu erreichen (RQ2.1)~\cite{autili2006conceptual}. Jeder Service hat eine Beschreibung, wobei zusätzliche QoS (Quality of Service) Attribute hinzugefügt werden. Alle verfügbaren Services werden in einer Service-Registry gespeichert. Services werden durch Serviceanfragen nachgefragt, wobei mehrere Anforderungen an den Service enthalten sind. Wenn ein Service aus der Registry passt wird dieser verwendet. Die Anfrage enthält zusätzliche Informationen wie verfügbare Speicher, Größe des Bildschirms oder der Typ der Netzwerkverbindung, was Informationen zum Kontext sind in dem der Service laufen wird (RQ2.3). Diese werden verwendet, um den nachgefragten Service an die Eigenschaften des Gerätes anzupassen, auf dem er laufen soll.
Der Ansatz wurde als Prototyp implementiert (RQ2.5). Es gibt weitere Forschungsmöglichkeiten bei der Verfeinerung des Kontextes (RQ3).

\subsection{Context-Aware Service Composition  A Methodology and a Case Study}
Bastida et al.~\cite{bastida2008context} zeigen einen Ansatz für die Entwicklung von  kontextsensitiven Servicekompositionen. Dabei ist das Ziel die Verbesserung von funktionalen und nicht funktionalen Anforderungen (RQ2.1) Dieser besteht aus sechs Schritten. Im ersten Schritt wird die Architekturspezifikation entwickelt, wobei auch Architekturalternativen identifiziert und bewertet werden, um die Systemanforderungen zu erfüllen. Im zweiten Schritt wird die funktionale Spezifikation ausgearbeitet, wobei alle Subfunktionalitäten identifiziert werden, um die endgültige Funktionalität der Komposition abzudecken~\cite{bastida2008context}. Im dritten Schritt wird die geeignete Komposition zur Designzeit identifiziert, sodass funktionale als auch nicht-funktionale Anforderungen erfüllt werden. Im vierten Schritt werden variable Punkte in der Komposition identifiziert. Dabei wird zur Designzeit identifiziert, welche Teile der Komposition sich ändern können und Anforderungen zur Anpassung definiert (RQ2.3)~\cite{bastida2008context}. Dabei ist das Ziel die Funktionalität der Komposition zur Laufzeit automatisch aufrechtzuerhalten (RQ2.2). Im fünften Schritt werden geeignet Services anhand der abstrakten Beschreibungen ausgewählt, um die endgültige ausführbare Komposition zu erhalten. Im letzten Schritt wird verifiziert, dass die ausgewählten Dienste ihre spezifizierte Funktionalität und QoS-Anforderungen erfüllen~\cite{bastida2008context}.
Die Anpassung der Komposition wird mithilfe von Regeln realisiert. Bei Variationspunkten werden die Regelbedingungen ausgeführt und ein konkreter Dienst für den Variationspunkt ausgewählt. Dies geschieht auf Basis von vorher definierten Varianten oder Alternativen für einen Variationspunkt.
Der Ansatz wurde mithilfe einer Fallstudie evaluiert (RQ2.5). Außerdem gibt es Verbesserungsmöglichkeiten des Ansatzes, um die möglichen Varianten dynamisch anzupassen. Zudem bestehen Forschungsmöglichkeiten bei der nachträglichen Anpassung der Komposition, bei der Teile durch eine Subkomposition ersetzt werden, um die gleiche Funktionalität zu erhalten (RQ3).

\subsection{Towards Context-aware Semantic Web Service Discovery through Conceptual Situation Spaces}
Um Kontextanpassungsfähigkeit zu erreichen, schlagen Dietze et al.~\cite{dietze2008towards} \textit{Conceptual Situation Spaces} (CSS) vor, welche die Beschreibung von situationsabhängigen Charakteristiken in geometrischen Vektorräumen erlauben (RQ2.3). Das Ziel ist die Verbesserung von funktionalen und nicht funktionalen Anforderungen (RQ2.1). Dabei beschreibt ein CSS einen bestimmten Kontext für eine bestimmte Situation. Die CSS sind dabei auf semantische Web-Services (SWS) ausgerichtet. Die semantische Ähnlichkeit zwischen Situationen wird dabei in Form ihrer euklidischen Entfernung innerhalb eines CSS berechnet~\cite{dietze2008towards}. Dabei werden die am besten geeigneten Ressourcen, wie Dienste oder Daten, automatisch basierend auf der semantischen Ähnlichkeit identifiziert (RQ2.2). Dieser werden aufgrund einer vorgegeben realen Situation aus Ressourcenbeschreibungen ausgewählt~\cite{dietze2008towards}.
Der Ansatz wurde in einem Prototyp implementiert, mit dem die Machbarkeit nachgewiesen wurde (RQ2.5).
Es werden weitere Forschungsmöglichkeiten bei der Priorisierung der verschiedenen Dimensionen eines Vektorraumes genannt, um den Ressourcenallokationsprozess besser auf die Vorlieben des Benutzers anzupassen. Außerdem soll die Beschreibung weiterer relevanter Kontextinformationen durch das CSS-Modell ermöglicht werden (RQ3).

\subsection{Context aware service composition}
Vukovic~\cite{vukovic2007context} präsentiert einen Ansatz, um kontextsensitive Anwendungen zu entwickeln. Dabei werden die Anwendungen als dynamische Zusammensetzung von Diensten betrachtet. Ändert sich der Kontext, in dem die Anwendung läuft, kann dies zu einer dynamischen Anpassung der Zusammensetzung führen (RQ2.2). Dabei wird der aktuelle Kontext des Benutzers der Anwendung verwendet (RQ2.3). Fragt der Nutzer einen Service nach, wird in der Anfrage zum einen eine Beschreibung der Aufgabe übermittelt und zum anderen kontextabhängige Parameter, wie das Gerät von dem die Anfrage kommt, oder die Information das der Nutzer sich gerade auf einer Straße mit einem Fahrzeug bewegt (RQ2.3).
Der Ansatz zielt auch auf Ausfälle von einzelnen Diensten ab. Tritt ein Fehler bei der Erstellung der Service-Zusammensetzung auf, wird versucht die Anfrage in eine Alternative Zusammensetzung zu ändern.
Änderungen von Kontext oder Ausfälle werden auch während der Ausführung überwacht und die Zusammensetzung der Services wird bei Bedarf entsprechend angepasst.
Das Ziel der Anpassung sind folglich die funktionale sowie nicht funktionale Anforderungen (RQ2.1).
Um Ausfälle zu kompensieren und den Kontext zu erfassen wird ein System namens GoalMorph vorgestellt, welches bei Ausfällen auf alternative Zusammensetzungen wechselt (RQ2.5). Der Ansatz beschreibt die Implementierung eines Frameworks für kontextsensitive Service Kompositionen. Der Ansatz wurde als Prototyp implementiert (RQ2.5).
Als weitere Forschungsmöglichkeiten wird eine benutzerdefinierte Anpassung an die Wünsche des Benutzers, eine Priorisierung der Anfragen sowie eine Verbesserung der QoS. Zudem werden Datenschutz, Sicherheit und Vertrauen als weitere Forschungsmöglichkeiten in Umgebungen mit Services von verschiedenen Anbietern genannt, da Kontextinformationen von Nutzern ähnlich wie die Nutzerdaten oft sensibel sind (RQ3).

\subsection{Behavioural Self-Adaptation of Services in Ubiquitous Computing Environments}
Camara, Canal und Salaün~\cite{camara2009behavioural} stellen einen Ansatz zur Selbstanpassung von Services vor. Dabei wird sowohl die Zusammensetzung der Services zur Designzeit als auch die dynamische Anpassung der Zusammensetzung zur Laufzeit thematisiert.
Dabei zielt der Ansatz auf das Protokoll oder die Verhaltensregeln ab. Dazu schlagen die Autoren ein Service Schnittstellenmodell vor, dass sowohl die Signatur als auch das Verhalten beinhaltet.
Der Ansatz zur Anpassung ist ein dynamisches System, bei dem sich Dienste zur Laufzeit dynamisch an- und abmelden können. Der Ansatz ist vollständig automatisiert und wird durch eine Anforderungsbeschreibung vom Benutzer gesteuert (RQ2.2, RQ2.3)~\cite{camara2009behavioural}. Wenn ein neuer Service der Choreography beitritt oder sie verlässt laufen drei Aufgaben ab. Im ersten Schritt wird ein Vektor aus mehreren Serviceschnittstellen durch abstrakte Operationssignaturen instanziiert~\cite{camara2009behavioural}. Im zweiten Schritt wird eine Liste von stabilen Zuständen ermittelt, welche Dienstprotokolle und Vektoren enthalten. Ein stabiler Zustand des Systems ist einer, bei dem sich jeder der Dienste im System in einem stabilen Zustand befindet~\cite{camara2009behavioural}. Dienste können nur in solch einem stabilen Zustand der Service-Zusammensetzung hinzugefügt oder entfernt werden. Nach den ersten beiden Schritten wird die Ausführung des Systems gestartet. Dazu wird eine Laufzeit-Engine vorgestellt, die solch ein System ausführt.
Zur Selbstanpassung werden Regeln vorgeben, die beispielsweise die Sicherheit betreffen. Dort wird beispielsweise definiert, was nicht passieren darf, wenn ein Service mit dem restlichen System kommuniziert.
Zudem werden sogenannte Lebendigkeitseigenschaften definiert, was passieren soll, wenn das System mit dem Rest interagiert~\cite{camara2009behavioural}. Diese gewährleisten, dass der Ablauf eines Services an einem gewünschten Punkt gehalten wird. Die Anpassung der Service Choreographie durch Hinzufügen oder Entfernen von Services im Ablauf geschieht immer unter Einhaltung der deklarierten Regeln.
Der Ansatz zielt auf die Verbesserung von funktionalen und nicht funktionalen Anforderungen ab (RQ2.1)
Von dem vorgeschlagenen Ansatz existiert kein Prototyp, wobei die Entwicklung eines solchen in den weiteren Forschungsmöglichkeiten thematisiert wird (RQ2.5, RQ3).

\subsection{Context-aware Adaptive Service Mashups}
Dorn et al. ~\cite{dorn2009context} stellen einen Ansatz zur kontextabhängigen Selbstanpassung von Service Kompositionen vor, bei dem die Fähigkeiten der Services mit den ständig aktualisierten Anforderungen an die Fähigkeiten abgeglichen werden. Dabei ist das Ziel die Verbesserung von funktionalen Anforderungen (RQ2.1). Der Anpassungsprozess besteht aus zwei Phasen, wobei in der ersten Phase Kontextänderungen und der betroffene Teil der Service Zusammensetzung ermittelt wird. Eine Konfiguration wird neu bewertet, wenn Anforderungsregeln durch die Kontextänderung betroffen sind. Die Anforderungsregeln generieren neue Fähigkeitseinschränkungen und eine Neuanpassung wird gestartet, wenn die bestehende Service Konfiguration nicht mehr mit den Einschränkungen übereinstimmt~\cite{dorn2009context}. Mit den Anforderungen wird eine Liste von Services zusammengestellt, deren Fähigkeiten die neuen Anforderungen am besten erfüllen. Die Services werden nach Gruppen geordnet und die Services jeder Gruppe bewertet, aus denen der Entwickler dann die beste Kombination auswählt. Der Ansatz ist folglich halbautomatisch (RQ2.2). Dies soll durch zukünftige Arbeiten durch automatisches generieren der Anforderungen automatisiert werden (RQ3).
Als Kontext zur Anpassung wird die Reputation der Services, die Organisationen die Service-Zusammensetzung verwenden und der physische Ort verwendet (RQ2.3).
Der Ansatz wurde zudem mit einer Fallstudie evaluiert (RQ2.5).

\subsection{Towards Context-Aware Adaptable Web Services}
Keidl und Kemper~\cite{keidl2004towards} zeigen einen Ansatz für ein Kontext-Framework für die Entwicklung und Bereitstellung von kontextabhängigen und anpassbaren Web-Services. Dabei ist das Ziel die Anpassung von funktionalen und nicht funktionalen Anforderungen (RQ2.1)
Der verwendete Kontext beinhaltet alle Informationen über den Client eines Web-Services, die vom Web-Service verwendet werden können, um die Ausführung und Ausgabe anzupassen und dem Client ein individuelles und personalisiertes Verhalten zu ermöglichen (RQ2.3)~\cite{keidl2004towards}. Der Kontext des Clients wird in der Anfrage an den Service übergeben, wobei der übergebene Kontext optional ist. Wenn der angefragte Service während seiner Ausführung einen anderen Service aufruft, wird der aktuelle Kontext automatisch in die Anfrage an den weiteren Service eingefügt (RQ2.2). Die Services nehmen sich jeweils die relevanten Kontextinformationen heraus. Der Kontext wird in der Serviceantwort zurückgeschickt.
Bei dem Ansatz wird zwischen explizitem und automatischen Kontext unterschieden. Expliziter Kontext sind Informationen, die in den Anfragen übergeben wurden und automatischer Kontext wird automatisch aus dem expliziten Kontext generiert.
Der Kontext wird durch verschiedenen Kontexttypen definiert. Es werden Typen wie Standort, Client, Verbraucher und Verbindungseinstellungen verwendet. Der Typ Client enthält beispielsweise Informationen über das Endgerät, wohingegen der Typ Verbraucher Informationen wie Name oder E-Mail enthält~\cite{keidl2004towards}. Der Typ Verbindungseinstellungen enthält wichtige Informationen zur Verbindung selbst.
Das Kontext-Framework und der Prototyp sind in Java implementiert und basieren auf Standards wie XML, SOAP, UDDI und WSDL (RQ2.5, RQ2.6)~\cite{keidl2004towards}.
Bei dem Ansatz bestehen weitere Forschungsmöglichkeiten bei zusätzlichen Kontexttypen und einer genaueren Verarbeitung des Kontextes (RQ3). Außerdem gibt es weiteren Forschungsbedarf bei den Sicherheitsrichtlinien, sodass festgelegt werden kann, welche Services Zugriff auf welchen Kontext haben und welche Dinge damit gemacht werden dürfen (RQ3).

\subsection{Context-Aware Service Composition in Pervasive Computing Environments}
Mokhtar et al.~\cite{mokhtar2005context} zeigen einen Ansatz zur dynamischen und kontextabhängigen Zusammenstellung von Services, der auf der Integration des Workflows basiert. Dabei ist das Ziel die Berücksichtigung des Kontextes des Benutzers und der kontextuellen Anforderungen der Dienste (RQ2.1)~\cite{mokhtar2005context}. Darüber hinaus ist das Ziel die Sicherstellung der QoS-Anforderungen des Benutzers. In dem Ansatz werden die Services mit der Ontology Web Language for Services (OWL-S) beschrieben (RQ2.6). Diese werden mit Kontextinformationen erweitert, wobei sich diese in hochrangige Kontextattribute und kontextuelle Voraussetzungen und Auswirkungen aufteilen. Kontextattribute sind beispielsweise der Standort oder physische Bedingungen. Kontextuelle Voraussetzungen sind dagegen Bedingungen, die für die Ausführung eines Services erfüllt sein müssen. Wie die Services werden auch die Benutzeraufgaben in OWL-S Prozessen beschrieben. Diese werden mit nicht-funktionalen Anforderungen, wie QoS-Anforderungen und Kontextbedingungen erweitert~\cite{mokhtar2005context}. Das Kontextmanagement basiert auf einem Kontextmanager, der bei Bedarf Kontextinformationen bereitstellt~\cite{mokhtar2005context}.
Zur Modellierung der OWL-S Prozesse wird eine formale Modellierung als endliche Automaten eingeführt.
Der Ansatz zur kontextabhängigen Servicekomposition ist zweistufig. Im ersten Schritt werden Services identifiziert, die atomare Prozesse bereitstellen und semantisch äquivalent in Bezug auf die bereitgestellten Fähigkeiten sind~\cite{mokhtar2005context}. Darüber hinaus werden dabei die kontextuellen Anforderungen mit den Kontextattributen der Dienste abgeglichen.
Im zweiten Schritt werden die Services integriert, die im ersten Schritt identifiziert wurden, um den Prozess des Benutzers dem geänderten Kontext anzupassen. Die Integration wird mit dem neu eingeführten Modell der endlichen Automaten durchgeführt.
Der Ansatz wurde zum Zeitpunkt des Papers noch nicht durch einen Prototypen validiert, aber es soll Teil künftiger Arbeiten werden, diesen zu bauen und den Ansatz zu validieren (RQ2.5, RQ3).

\subsection{Towards Context-Aware Composition of Web Services}
Luo et al.~\cite{luo2006towards} präsentieren ein Framework zur kontextabhängigen Komposition von Web-Services, wobei passende Services nach den Anforderungen des Benutzers gesucht, zusammengestellt und mit Petrinetzen auf Korrektheit überprüft werden. Der Ansatz unterstützt funktionale und nicht funktionale Anforderungen (RQ2.1). Zur Modellierung der Dienste wird OWL-S verwendet (RQ2.6). Angebotene Services werden von den Anbietern in einer Registry registriert. Für das automatische Suchen, Zusammenstellen, Aussuchen und Ausführen der Services werden Agents verwendet (RQ2.2).
Für das Annehmen der Anfragen sowie für das Antworten des Ergebnisses wird ein User Agent verwendet. Dieser sammelt Kontextinformationen über den Benutzer. Diese beinhalten beispielsweise die ID, den Ort, die Fähigkeiten des Endgeräts oder die Qualität des gewünschten Web-Services (RQ2.3). Außerdem speichert der User Agent die Präferenzen des Benutzers von bisherigen Anfragen.
Für das Suchen und Auswählen der Services ist der Broker Agent verantwortlich. Dieser extrahiert den Kontext aus der Anfrage und versucht den am besten passenden Service aus der Registry zu ermitteln. Gelingt ihm dies nicht exakt, wird die Anfrage in verschiedene Unterziele zerlegt, um annähernd passende Dienste zu finden~\cite{luo2006towards}. Dabei wird eine Liste von möglichen Kompositionen zusammengestellt. Nach dem Finden und Zusammenstellen der Services sucht der Broker Agent die beste Komposition aus. Dies geschieht auf Basis des aktuellen Kontextes und den Einschränkungen der Service Anbieter. Am Ende wird eine Ausführungssequenz generiert und die Informationen an den Service Execution Agent übergeben. Dieser ist für das Aufrufen der involvierten Services verantwortlich. Der Service Execution Agent besteht aus einem Prozess Verifizierer und einer Ausführungs-Engine. Der Prozess Verifizierer prüft mit Petrinetzen ob die Ausführungssequenz machbar ist und ob alle Services miteinander kompatibel sind~\cite{luo2006towards}. Die Ausführungs-Engine führt die verifizierten Services der Reihe nach aus. Nach dem Ausführen wird das Ergebnis an den Benutzer übermittelt und das Prozessmodel wird zusammen mit dem Kontext in dem Cache für zukünftige Anfragen gespeichert.
Das vorgeschlagene Modell mit den Petrinetzen wurde mit Hilfe eines Erreichbarkeitsbaumes validiert. Der Ansatz wurde in einem Projekt implementiert (RQ2.5). In zukünftigen Arbeiten soll die Anpassung der Service-Kompositionen zur Laufzeit untersucht werden, um beispielsweise auf Ausfälle von Services reagieren zu können (RQ3).

\subsection{A Conceptual Framework for Provisioning Context-aware Mobile Cloud Services}
La und Kim \cite{LK10} stellen ein Framework vor, das verteilte Services, die von mobilen Endgeräten eingebunden werden, kontextbasiert anpasst. Dabei wird sowohl die Auswahl die einzubindenden Services als auch dessen Anpassung bei Kontextänderungen beeinflusst. Das Framework bietet eine Entscheidungsfindung bei der Frage nach der notwendigen Anpassung bei einer Kontextänderung, indem zunächst der Unterschied zwischen altem und neuem Kontext analysiert und kategorisiert wird. Darauf basierend wird ein passender Adaptertyp aus einer vordefinierten Menge ausgewählt. Beispiele sind ein Interface-Adapter, der unterschiedliche Services mit der gleichen Funktionalität, die sich nur im Kontext wie etwa dem Land, für das sie gedacht sind, unterscheiden, bedienen kann sowie Reroute-Adapter, der verschiedene Instanzen desselben Services bedienen kann, zum Beispiel um die Latenz nach einem Ortswechsel zu verbessern.

Das Framework kann mit verschiedenen Kontexttypen umgehen (RQ2.3): Geräte-Kontext, Nutzer-Einstellungen, Situations-Kontext und Service-Kontext. Der Kontext wird dabei von mobilen Endgeräten wie Smartphones erhoben. Die Kategorien der Auswahl des passenden Adapters beinhalten sowohl funktionale als auch nicht-funktionale Anforderungen (RQ2.1). Die Anpassungen werden automatisch vorgenommen (RQ2.2). Das Framework zielt auf die Nutzung mit Services in Cloudumgebungen ab. Die dadurch mögliche Skalierung wird nicht eingeschränkt, aber auch nicht gefördert (RQ2.4). La und Kim \cite{LK10} haben das Framework in einer Case Study auf ein System angewendet und positive Ergebnisse erhalten (RQ2.5). Das Framework beruht auf verteilten Services, die von mobilen Endgeräten direkt eingebunden werden (RQ2.6).

\subsection{A context-aware adaptive web service composition framework}
Cao et al. \cite{CZZ15} nutzen ein Framework aus drei Teilen, um eine Web Service Choreography Kontext-basiert anzupassen. Das \textit{Web Service Composition Module} verwaltet die Choreography, das \textit{Context-Aware Module} prüft den Kontext auf Änderungen und das \textit{Adaptive Management Module} ändert basierend auf den festgestellten Kontextänderungen die Komposition der Services. Dazu wird mit Hilfe der Änderungen die am besten passenden Policy aus einer Policy-Bibliothek ausgewählt. Eine Policy gibt eine bestimmte Komposition der Services vor. Diese wird dann auf die Web Service Komposition angewendet.

Das Framework von Cao et al. \cite{CZZ15} nimmt ausschließlich Nicht-Funktionale Änderungen vor (RQ2.1). Es arbeitet automatisch (RQ2.2) und unterscheidet zwischen Service-, Physical- und User-Kontext. Diese Kategorien werden sehr weit gefasst, sie schränken den beachteten Kontext nicht ein. Es wird dabei nicht auf die Erhebungsmethode eingegangen (RQ2.3). Auf Skalierbarkeit wird nicht eingegangen (RQ2.4). Ein Prototyp wird von Cao et al. \cite{CZZ15} vorgestellt (RQ2.5). Das Framework nutzt eine auf BPEL aufbauende Web Service Choreography (RQ2.6).

\subsection{Context-Aware Service Adaptation: An Approach Based on Fuzzy Sets and Service Composition}
Madkour et al. \cite{MEM13} beschreiben sehr detailliert, wie Kontext formal dargestellt werden kann. Dazu werden einem relevanten Objekt, etwa dem Nutzer, verschiedene Verbindungen zu anderen Objekten zugewiesen. Eine Verbindung hat dabei eine Mächtigkeit. Zum Beispiel hat ein Nutzer eine Verbindung zu einem Ort-Objekt mit einer Mächtigkeit, die die Entfernung zu diesem Ort angibt. Die formale Beschreibung ermöglicht die gewünschten Anpassungen der Service Komposition, die sich aus der am besten zur Situation passenden Policy ergibt, ebenfalls formal darzustellen. Dies ist dann eine formal definierte Anforderung zur Änderung der Komposition. Daraus werden unter Einsatz verschiedener Methoden abstrakte Ziel-Kompositionen erstellt. Madkour et al. \cite{MEM13} nennen als Beispielmethode Artificial Intelligence Planning, was jedoch nicht weiter beschrieben ist. Eine abstrakte Ziel-Komposition kann dann auf die tatsächliche Service-Komposition angewendet werden.

Der Ansatz unterstützt Funktionale und Nicht-Funktionale Anforderungen (RQ2.1) und funktioniert automatisch (RQ2.2). Es werden der Nutzer- und Service-Kontext beachtet, wobei die Umgebung und Situation dem Nutzer zugeordnet wird und damit ebenfalls abgedeckt ist (RQ2.3). Skalierbarkeit wird nicht thematisiert (RQ2.4). Ein Prototyp ist nicht vorhanden, jedoch angedacht (RQ2.5). Der Ansatz beruht auf Service-Kompositionen, die als Choreography klassifiziert werden können (RQ2.6).

\subsection{PerCAS: An Approach to Enabling Dynamic and Personalized Adaptation for Context-Aware Services}
Yu et al. \cite{YHS12} stellen einen Model-basierten Ansatz zur Kontext-basierten Adaption von Web Service Kompositionen vor. Dazu wird die grundlegende Logik eines Service in einem \textit{Base Functionality Model} dargestellt. Die Kontext-abhängige Logik wird als \textit{Personalized Context-Awareness Logic Model} dargestellt. Ein solches Model besteht aus Regeln, die je aus einer Voraussetzung und einer daraus resultierenden Aktion bestehen sowie einem Typ. Regeln desselben Typs können zur Laufzeit dynamisch ausgetauscht werden. Dadurch kann jederzeit ein für einen Nutzer passendes Regel-Set erstellt und angepasst werden. \textit{Aspects} ordnen eine Regel einer Aktivität der grundlegenden Logik zu, indem diese eine Aktivität und einen Regel-Typ enthalten. Wird bei Durchführung einer Aktivität eine Voraussetzung einer zugeordneten Regel erfüllt, wird die Aktion dieser Regel ausgeführt.

Regel dieses Ansatzes können für Funktionale und Nicht-Funktionale Anforderungen erstellt werden (RQ2.1). Die Auswahl der Regeln erfolgt manuell oder durch ein zusätzliches System (RQ2.2). Es wird ausschließlich Nutzer-Kontext genutzt (RQ2.3). Aus Skalierbarkeit wird nicht eingegangen (RQ2.4), ein Prototyp ist vorhanden (RQ2.5). Der Ansatz nutzt eine Web Service Choreography (RQ2.6).

\subsection{Context-aware autonomous web services in software product lines}
Alférez und Pelechano \cite{AP11} haben in ihrem Ansatz detailliert beschrieben, wie mit Hilfe von \textit{Software Product Lines (SPL)} Web Services automatisch Kontext-basiert angepasst werden können. Der Ansatz beruht auf \textit{Dynamic Software Product Lines} \cite{HHP08}, was es ermöglicht, SPL-basierte Services dynamisch zu laden. Dazu werden bereits bei der Entwicklung Punkte in der Ausführsequenz festgelegt, an denen die Software je nach definierten Parametern unterschiedlich fortgeführt wird. Diese Punkte werden genutzt, um Änderungen an der Service-Komposition einzuleiten. Der Ansatz setzt externe Services als Kontextquelle voraus, die die genannten Parameter definieren.

Der Ansatz unterstützt je nach Kontextquelle Funktionale und Nichtfunktionale Anforderungen (RQ2.1). Der Ansatz funktioniert selbstständig (RQ2.2), er geht nicht auf Skalierbarkeit ein (RQ2.4). Der verwendete Kontext hängt komplett von externen Services ab (RQ2.3) und beruht auf einer Web Service Choreography (RQ2.6). Eine Umsetzung ist als Case Study vorhanden (RQ2.5).

\subsection{Context-aware pervasive service composition and its implementation}
Zhou et al. \cite{ZGP11} beschreiben sehr abstrakt, welche Schritte für eine Kontext-basierte Anpassung von Service-Kompositionen notwendig sind. Dies sind (i) das Sammeln von Kontext-Informationen, (ii) die Beschreibung vorhandener Services, (iii) das Untersuchen der modellierten Services gegen die gesammelten Daten und (iv) die Anpassung der Service-Komposition aufgrund der Ergebnisse des dritten Schritts. Der letzte Schritt kann die verwendeten Services ändern, die Logik eines Service ändern und Hilfsfunktionen ändern. Der Ansatz beschreibt nicht, wie all dies umgesetzt werden kann, sondern bietet lediglich eine beispielhafte Umsetzung (RQ2.5).

Schritt (ii) erfolgt manuell, die anderen automatisch (RQ2.2). Der Ansatz ist gedacht für Service Choreographies (RQ2.6). Da der Ansatz keine Funktionsweise, sondern lediglich eine abstrakte Theorie beinhaltet, sind RQ2.1, RQ2.3 und RQ2.4 nicht zutreffend.

\subsection{Context-aware pervasive service composition and its implementation}
Romero et al. \cite{RRS10} stellen eine Plattform für Kontext-basierte Web Services vor. Diese nutzt mehrere Frameworks für verschiedene Aufgaben und kombiniert diese. Das \textit{Context Entities Composition and Sharing (COSMOS)} Framework \cite{RCS08} ist dafür verantwortlich, Kontextinformationen zu sammeln und bereitzustellen. Dafür werden alle für einen Service notwendigen Kontextinformationen in Policies modelliert. Diese wiederrum werden in möglichst kleine Teile geteilt und hierarchisch gruppiert. Jedes Teil ist für eine Information verantwortlich und sammelt diese mit Hilfe von Operatoren. Standardoperatoren, etwa zum Auslesen von Betriebssystemdaten und Benutzereinstellungen, sind im Framework enthalten. Alle weiteren müssen selbst implementiert werden. Die Kontext Policies werden mit Hilfe des im Ansatz vorgestellten \textit{SPACES} Framework \cite{SMF09} in einer verteilten Umgebung auf die Services verteilt.

Die Plattform nutzt Kontrollzyklen bestehend aus der Kontextbeschaffung, der Analyse des Kontexts, des Planens der Anpassungen an den Services und der Durchführung der Anpassungen. Schritt 1 wird mit \textit{COSMOS} durchgeführt. Für Schritt zwei und drei werden servicespezifische Entscheidungsalgorithmen vorausgesetzt. Ein Service muss einen solchen Algorithmus bereitstellen. Mit Hilfe eines Anpassungsplans kann die Plattform \textit{FraSCAti} genutzt werden, die Services zur Laufzeit ändern, neu ausrollen und neu binden kann.

Die Plattform von Romero et al. \cite{RRS10} macht keine Einschränkungen hinsichtlich der Anforderungen (RQ2.1) und funktioniert automatisch (RQ2.2). Es kann jeder beliebige Kontext genutzt werden, sofern die notwendigen Kontextoperatoren hinzugefügt werden (RQ2.3). Die Plattform macht keine Einschränkungen hinsichtlich der Skalierbarkeit (RQ2.4). Ein Beispiel ist vorgestellt (RQ2.5). Die Plattform beruht vollständig auf Web Service Choreographies (RQ2.6).

%Senaits Zeug Beginn
\subsection{Service Composition in a Secure Agent-Based Architecture}
Der Ansatz von Bharadwaj et al.~\cite{bharadwaj2005service} beschreibt eine agentenbasierte, situationsbezogene und überlebensfähige Architektur für die Ermittlung und Zusammenstellung von Webdiensten (RQ1). Dabei stützt sich die Architektur auf dezentrale situationsbezogene Umgebungen (situation-aware ambients (SAAs)), bei denen es sich um autonome Agenten handelt~\cite{bharadwaj2005service}. Diese verarbeiten Informationen über die aktuelle Situation, erkennen und stellen Dienste syntaktisch zusammen, um sich an verändernde Situationen anzupassen (RQ2.3)~\cite{bharadwaj2005service}.
Das Ziel ist dabei die Verbesserungen von QoS (Quality of Service), wie beispielsweise Sicherheit, Aktualität und Verfügbarkeit (RQ 2.1). Die Schwierigkeit dies bei verteilten Services zu erreichen, soll mit diesem Ansatz verbessert werden. Dazu müssen die verschiedenen Services dynamisch migrieren können. Zur Erhöhung der Fehlertoleranz und der Verfügbarkeit ist das Ziel die Erkennung der Situation, um auf die geänderten Bedingungen reagieren zu können, was durch eine Anpassung der Service Choreografie gelöst wird.

Dazu werden Services und Ressourcen deklarativ beschrieben.  
Dabei werden Formeln eingesetzt, die arithmetische Einschränkungen beinhalten, um Echtzeitfristen einzuhalten und Vorgehensweisen für Zeit- und Ortsbeschränkungen festzulegen (RQ2.6)~\cite{bharadwaj2005service}. Die Logik besitzt dazu ein vollständiges Beweissystem~\cite{bharadwaj2005service}. Hierbei werden funktionale Anforderungen einschließlich der QoS (Quality of Service) Einschränkungen, wie beispielsweise Zugriffskontrolle und Sicherheit berücksichtigt, die durch den Benutzer spezifiziert werden (RQ2.1). Die Logik, um das Modell zu generieren welches eine Menge von SAAs umfasst, wird von einer deduktiven Engine implementiert~\cite{bharadwaj2005service}. Um die Anforderungen erfüllen zu können wird ein Model aus den Beschreibungen der Services generiert. Zur Erreichung der QoS wird in einem solchen Plan festgehalten wie die Agenten Hosts betreten, verlassen oder mit anderen SAAs kommunizieren und zusammenarbeiten (RQ2.2)~\cite{bharadwaj2005service}. Aktuell basiert die Beweis Engine, welche die SAAs ableitet, auf Prolog und es wird an einer schnelleren gearbeitet (RQ 2.5). Zudem wird an neuen Techniken für die Erkennung von Services und Fehlern gearbeitet (RQ3).

\subsection{Middleware for ubiquitous context-awareness}
Die Arbeit von da Rocha et al. \cite{rocha2008middleware} diskutiert die Herausforderungen und Kompromisse von einer Middleware-Implementierung, die allgegenwärtiges Kontextbewusstsein unterstützt, wie beispielsweise einem Szenario, in dem sich kontext-sensitive Anwendungen durch Netzwerkumgebungen bewegen können ohne in ihrer kontextbasierten Interaktion unterbrochen werden zu müssen. Es wird ein Middleware-Ansatz vorgestellt, das auf den Kontextdomänenmodellen basiert, welche einige Anforderungen des besagten Szenarios erfüllen (RQ1). Ein kontext-sensitives System muss jederzeit einen effizienten Zugriff auf die genutzten Kontextdaten sicherstellen, auch wenn das System stark skaliert wird (RQ2.3). Um Skalierbarkeit, Sicherheit und Performance zu gewährleisten schlägt Rocha et al. \cite{rocha2008middleware} vor, den Einsatzzweck von Kontextdaten zu limitieren (RQ2.4). Die vorgeschlagene Architektur in dieser Arbeit wurde mittels Java implementiert (RQ2.5) Diese besteht aus \textit{Context Management Nodes} (CMN). Ein CMN wird aus mehreren Bestandteilen zusammengesetzt: Ein \textit{Event Service} sorgt für asynchrone Kommunikation, ein \textit{Type System Manager} hält die sich ändernden Kontextquellen und ein \textit{Context Repository} die daraus gewonnene Kontextdaten. Außerdem sorgt ein \textit{Context Delivery Service} dafür, dass von allen Kontextquellen, die als Entities gehalten werden, die Daten gesammelt werden. Das System arbeitet IP-basiert, wofür das \textit{Service Location Protokoll} übernommen wurde, um den CMNs zu ermöglichen, Mobilgeräte, die sich in ein Netzwerk verbinden, zu erkennen. Als Basisinfrastruktur für den Zugriff auf unformatierte Kontextdaten und für symbolische Standorte wird eine vorhandene Kontext-Provisioning-Middleware (MoCA) \cite{SER04} verwendet (RQ2.6). 

\subsection{Situation-awareness for adaptive coordination in service-based systems}

Die Veröffentlichung von Yau et al. \cite{huang2005situationaware} behandelt die Thematik der servicebasierten Systeme. Ein solches System hat viele Anwendungsszenarien, einschließlich kollaborativer Forschung und Entwicklung, E-Business, Gesundheitswesen, Umweltkontrollen, militärische Anwendungen und Heimatschutz. Daher ist für diese Systeme eine Dienstkoordinierung erforderlich, um verteilte Aktivitäten zu koordinieren. Um eine adaptive Servicekoordination unter sich ändernden Umgebungs- und Arbeitsbedingungen zu erreichen, ist deshalb ein situationsbewusstes System erforderlich. Diese Arbeit stellt ein Model für die Anforderungen an die Informationswahrnehmung (situation-awareness, SAW) in dienstbasierten Systemen vor (RQ1). Dieses Modell dient als Basismodell und es werden SAW-Agenten entwickelt, um Situationsbewusstsein  und adaptive Koordination in servicebasierten Systemen zu erhalten (RQ2.6). Die Integration von SAW für adaptive Koordination in servicebasierten Systemen besteht aus zwei Hauptteilen: 1) Das Modellieren und das Spezifizieren von SAW für adaptive Dienstkoordination sowie 2) Der Entwicklung von SAW-Agenten für adaptive Servicekoordination. Einen Missionsplaner oder ein ähnliches System, welches vom Benutzer definierte Ziele akzeptiert und auf Basis der verfügbaren Services und der aktuellen Situation Ausführungspläne erstellt, setzt der Ansatz voraus. Ein generierter Ausführungsplan ist eine Reihe von Servicekompositionen, die ausgeführt werden müssen, um das Gesamtziel zu erreichen (RQ2.3). Jeder Schritt in einem Ausführungsplan kann Abhängigkeiten zu Situationen aufweisen, wodurch dieser Schritt nur ausgeführt werden kann, wenn die erforderliche Situation und damit ein bestimmter Kontext eintrifft (RQ2.2). In zukünftigen Arbeiten sind Nicht-Funktionale Anforderungen vorgesehen (RQ2.1), wie Sicherheit und Echtzeit in Servicekoordination und Softwareentwicklung (RQ3).  

%Senaits Zeug Ende

\section{Verwandte Arbeiten}\label{s:verwandteArbeiten}
Aktuell gibt es noch keine systematischen Literaturrecherchen zu kontextabhängigen Anpassung von Choreographien.
Eine Service Choreograpy wird in diesem Zusammenhang als kontextsensitiv bezeichnet, wenn sie \glqq den Kontext verwendet, um dem Benutzer relevante Informationen und/oder Dienste bereitzustellen, wenn die Relevanz von der Aufgabe des Benutzers abhängt.\grqq~\cite[S.~2]{dey2001understanding}

Leite et al. \cite{leite2013systematic} haben eine \textit{Systematic Literature Review} zu \textit{Service Choreography Adaption} erstellt. Die gefundenen Arbeiten werden nach ihrer verwendeten Adaptionsstrategie kategorisiert. Ein Drittel der ausgewählten Arbeiten werden der Model-basierten Kategorie zugeordnet. Dabei werden vorzunehmende Änderungen in Modellen der zu ändernden Anwendung vorgenommen und damit auf einer höheren Abstraktionsebene als auf der Code-Ebene. Eine Änderung an einem Model ist auch gleichzeitig der Auslöser für eine Adaption, da davon ausgegangen wird, dass Nicht-Informatiker die Änderungen je nach Bedarf selbstständig manuell vornehmen können. Jeweils ein Sechstel der Arbeiten werden in die Kategorien Messung-basiert, Multi-Agent-Systeme und Formale Methoden eingeordnet. Messung-basierte Ansätze werden durch das Überschreiten von Grenzwerten ausgelöst. Die Auswahl der neuen Komposition geschieht durch Berechnung der zugrundeliegenden Werte der Alternativ-Kompositionen und einem Vergleich dieser Werte. Multi-Agent-Systeme bestehen aus unabhängig agierenden Agenten, die wenn nötig Nachrichten und damit Informationen austauschen. Die Agenten passen sich dabei selbst an, je nach Informationen, die ihnen vorliegen. Dazu wird häufig \textit{Artificial Intelligence} genutzt. Die Kategorie der Formalen Methoden enthält Arbeiten, die auf Prozesskalkülen oder Endlichen Zustandsautomaten beruhen. Diese erlauben die Beschreibung der Services auf höheren Abstraktionsebenen. Änderungen werden ähnlich den Model-basierten Arbeiten an den Abbildungen vorgenommen und daraus automatisiert die Adaption der Services abgeleitet. Die beiden verbleibenden Kategorien mit jeweils einem Zwölftel der Arbeiten sind Semantische Ansätze und Proxy-Layer Ansätze. Bei Semantischen Ansätzen werden gezielt die logischen Relationen zwischen Services genutzt. Die Choreographie Komposition wird so angepasst, dass eine Kommunikation zwischen den Services möglich ist, sollte dies aufgrund von Änderungen durch Dritte nicht mehr möglich sein. Proxy-Layer Ansätze beruhen darauf, die Kommunikation zwischen Services zu ändern, anstatt die Komposition tatsächlich zu ändern. Dazu wird der Proxy zwischengeschaltet, alle Nachrichten laufen über den Proxy und werden wenn nötig abgeändert.
Das \textit{Systematic Literature Review} von Leite et al. \cite{leite2013systematic} gibt einen Überblick zu Adaptionsstrategien von Service Choreographien. Der Grund einer Adaption, etwa aus dem Kontext heraus, wird dabei nicht beachtet.
Die SLR von Leite et al.~\cite{leite2013systematic} kommt der vorliegenden SLR am nächsten wobei der Kontext dort nicht behandelt wurde.

Truong et al.~\cite{truong2009survey} haben 2009 eine Erhebung zur Analyse aktueller Techniken und Verfahren für kontextabhängige Web-Service Systeme durchgeführt. Dabei wird nicht explizit nach Service Choreographien gesucht, sondern allgemein nach kontextabhängigen Web-Service Systemen, wodurch sich diese SLR von der Erhebung von Truong et al.~\cite{truong2009survey} abgrenzt. Die Arbeit fokussiert sich dabei auf kontextbezogene Techniken und analysiert wie diese auf Systeme von Web-Services angewendet werden.

Di Nito et al.~\cite{di2008journey} diskutieren in ihrer Arbeit die immer größer werdende Bedeutung von serviceorientierten Anwendungen die sich selbst anpassen können. Dabei werden verschiedene Konzepte vorgestellt, mit denen dynamische, anpassungsfähige und serviceorientierte Systeme realisiert werden können. Dabei werden aktuelle Studien und Technologien zu den Bereichen \textit{Business Process Management}, Service Komposition und Koordination, Service Infrastruktur, Analyse Design und Entwicklung, Servicequalität, Anpassung zur Laufzeit und Kontextsensitivität zusammengefasst. Außerdem werden offene Fragen und Herausforderungen diskutiert.

\section{Zusammenfassung}\label{s:zusammenfassung}
Eine Service-Choreographie ist eine Beschreibung von mehreren Services, welche sich dezentral organisieren, indem die einzelnen Services durch Austauschen von Nachrichten, Interaktionsregeln und Vereinbarungen miteiander kommunizieren. Um sich ändernden Anforderungen anzupassen und die Servicequalität zu erhöhen werden Strategien zur Anpassung von Service-Choreographien benötigt. Dazu kann der Kontext einer Situation verwendet werden, zu der sich die Choreographie anpassen soll. Zu dem Kontext gehören alle Informationen, die in Zusammenhang mit der aktuellen Situation oder Handlung sind, wie beispielsweise Ort oder Informationen über den Nutzer.

Diese Arbeit fasst den aktuellen Forschungsstand zur kontextabhängigen Anpassung von Service-Choreographien zusammen. Das Vorgehen der Systematischen Literaturrecherche ist protokolliert, was die Reproduzierbarkeit sicherstellt. Die Literaturrecherche besteht aus einer dreiteiligen Suchanfragen, welche automatisch auf den Suchmaschinen Google Scholar, Springer Link und IEEEXplore ausgeführt wurden. Die Ergebnisse sind automatisch vorsortiert und manuell nach Ein- und Ausschlusskriterien nachsortiert, woraus insgesamt 22
themenrelevante Ergebnisse resultieren. Für jedes Ergebnis ist zusammengefasst, welche Adaptionsstrategie in dem Ansatz verwendet wird. Des Weiteren ist dargelegt, wie die Adaption durchgeführt wird. Dabei ist aus den Ergebnissen Informationen zum Ziel der Anpassung, der erforderlicher Eingriffsgrad, Aspekte zur Kontextsensitivität, Auswirkungen auf die Skalierbarkeit, bestehende Implementierungen sowie zugrundeliegende Modelle zusammengefasst. Zudem sind die bestehenden Limitationen zu jedem Ergebnis aufgeführt.

Die vorgestellten Ansätze zeigen, dass häufig eigene Module zum Sammeln, zur Analyse und zum Bearbeitung des Kontexts genutzt werden. Außerdem ist festzustellen, dass der auf Änderungen zu überwachende Kontext und die daraus folgenden Änderungsmöglichkeiten oft vordefiniert werden müssen. Die Ansätze sind daher meist nicht in der Lage, selbstständig aus einer Situation zu lernen und Adaptionen vorzunehmen. Zum Teil wird daher versucht, die dafür notwendigen Informationen vor der Ausführung als Model darzustellen, sodass diese von Experten des Fachgebiets, für das die Anwendung gedacht ist, selbstständig formuliert werden können.

Eine Kategorisierung der Ergebnisse ist schwierig. Zwar gibt es Gemeinsamkeiten zwischen den Ansätzen, diese sind aber so generisch, dass sie auf alle ausgewählten Ansätze zutreffen. Bei detaillierterer Betrachtung unterscheiden sich die Ansätze allerdings so stark, dass eine Einteilung in größere Kategorien nicht möglich ist.

Eine Schwierigkeit bei der Endauswahl der Ansätze zur kontextabhängigen Anpassung von Service-Choreographien ist die ungenaue Bezeichnung und Abgrenzung der verwendeten Service-Komposition in einem Ansatz. Oftmals wird eine Choreographie nicht als solche bezeichnet, auch wenn sie vorhanden ist.

In Zukunft kann die systematische Literaturrecherche zur kontextabhängigen Anpassung von Service-Choreographien erneut durchgeführt werden, um neue Ansätze zu finden. Dies ist durch das automatisierte Skript und das Protokoll möglich. Zudem können die Suchanfragen weiter verfeinert werden, indem neue Suchwörter zu den bestehenden hinzugefügt werden.

%% use section* for acknowledgment
%\ifCLASSOPTIONcompsoc
%  % The Computer Society usually uses the plural form
%  \section*{Acknowledgments}
%\else
%  % regular IEEE prefers the singular form
%  \section*{Acknowledgment}
%\fi
%
%
%The authors would like to thank...





% trigger a \newpage just before the given reference
% number - used to balance the columns on the last page
% adjust value as needed - may need to be readjusted if
% the document is modified later
%\IEEEtriggeratref{8}
% The "triggered" command can be changed if desired:
%\IEEEtriggercmd{\enlargethispage{-5in}}

% references section

% can use a bibliography generated by BibTeX as a .bbl file
% BibTeX documentation can be easily obtained at:
% http://mirror.ctan.org/biblio/bibtex/contrib/doc/
% The IEEEtran BibTeX style support page is at:
% http://www.michaelshell.org/tex/ieeetran/bibtex/
%\bibliographystyle{IEEEtran}
% argument is your BibTeX string definitions and bibliography database(s)
%\bibliography{IEEEabrv,../Bibliography/IEEEfull}
%
% <OR> manually copy in the resultant .bbl file


\bibliographystyle{Bibliography/IEEEtran}
\bibliography{Bibliography/IEEEabrv,Bibliography/bibliography}

\clearpage
\newgeometry{margin=6cm}
\onecolumn

% Die eidesstattliche Erklärung mit Unterschrift
\section*{Erklärung}

Ich versichere, diese Arbeit selbstständig verfasst zu haben.
Ich habe keine anderen als die angegebenen Quellen benutzt und alle wörtlich oder sinngemäß aus anderen Werken übernommene Aussagen als solche gekennzeichnet.
Weder diese Arbeit noch wesentliche Teile daraus waren bisher Gegenstand eines anderen Prüfungsverfahrens.
Ich habe diese Arbeit bisher weder teilweise noch vollständig veröffentlicht.
Das elektronische Exemplar stimmt mit allen eingereichten Exemplaren überein.

\vspace{4cm}

\hrule
\vspace{0.2cm}
Ort, Datum \hfill Unterschrift \hspace{2cm}
% that's all folks

\clearpage

\onecolumn
% Die eidesstattliche Erklärung mit Unterschrift
\section*{Erklärung}

Ich versichere, diese Arbeit selbstständig verfasst zu haben.
Ich habe keine anderen als die angegebenen Quellen benutzt und alle wörtlich oder sinngemäß aus anderen Werken übernommene Aussagen als solche gekennzeichnet.
Weder diese Arbeit noch wesentliche Teile daraus waren bisher Gegenstand eines anderen Prüfungsverfahrens.
Ich habe diese Arbeit bisher weder teilweise noch vollständig veröffentlicht.
Das elektronische Exemplar stimmt mit allen eingereichten Exemplaren überein.

\vspace{4cm}

\hrule
\vspace{0.2cm}
Ort, Datum \hfill Unterschrift \hspace{2cm}


%\clearpage
%\onecolumn
% Die eidesstattliche Erklärung mit Unterschrift
%\section*{Erklärung}
%
%Ich versichere, diese Arbeit selbstständig verfasst zu haben.
%Ich habe keine anderen als die angegebenen Quellen benutzt und alle wörtlich oder sinngemäß aus anderen Werken übernommene Aussagen als solche gekennzeichnet.
%Weder diese Arbeit noch wesentliche Teile daraus waren bisher Gegenstand eines anderen Prüfungsverfahrens.
%Ich habe diese Arbeit bisher weder teilweise noch vollständig veröffentlicht.
%Das elektronische Exemplar stimmt mit allen eingereichten Exemplaren überein.
%
%\vspace{4cm}
%Gingen an der Fils, 9.8.2019\\              
%\includegraphics[width=3cm, right]{jpeg/Unterschrift.jpeg}
%\hrule
%\vspace{0.2cm}
%Ort, Datum \hfill Unterschrift \hspace{2cm}

\clearpage

\onecolumn
% Die eidesstattliche Erklärung mit Unterschrift
\section*{Erklärung}

Ich versichere, diese Arbeit selbstständig verfasst zu haben.
Ich habe keine anderen als die angegebenen Quellen benutzt und alle wörtlich oder sinngemäß aus anderen Werken übernommene Aussagen als solche gekennzeichnet.
Weder diese Arbeit noch wesentliche Teile daraus waren bisher Gegenstand eines anderen Prüfungsverfahrens.
Ich habe diese Arbeit bisher weder teilweise noch vollständig veröffentlicht.
Das elektronische Exemplar stimmt mit allen eingereichten Exemplaren überein.

\vspace{4cm}

\hrule
\vspace{0.2cm}
Ort, Datum \hfill Unterschrift \hspace{2cm}


\end{document}

